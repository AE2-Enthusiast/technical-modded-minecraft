\documentclass{article}

\title{Mana Generation}
\author{AE2 Enthusiast}
\date{\today}

\begin{document}
\maketitle
\tableofcontents

\section{Flowers}
Generating flora generate mana, obviously. Most flowers have internal caps to
how much mana they can produce. Typically, flowers simply have a hard cap to how
much fuel they can eat or how much mana they make in a second, though some, like
the Dandelifeon, has more obscure limitations. In either case, the limit imposed
is lower than the maximum possible and those flowers can be ignored.

The four flowers that have no real cap to their mana production are the
Entropynnium, Rosa Arcana, Shulk-Me-Not, and Spectrolus. Technically, these
flowers are capped by the fact that they cannot produce more mana then they
store, but that capped mana rate is much faster then the maximum and doesn't
matter

\subsection{Entropynnum}
The Entropynnium is simple. It eats TNT explosions and produces mana. The
Entropynnium can eat one TNT every tick and produces it's entire internal mana
buffer (6500) each time. The Entropynnium has a square range of 12 blocks in
each axis, or a 25x25x25 cube.

An important thing to note is that the Entropynnium is not actually eating TNT
explosions. Instead, it is checking for Primed TNT entities nearby with a fuse
timer of 1 that aren't in water. The Entropynnium then deletes the Primed TNT,
produces a fake explosion effect, and produces mana. This is only important
because it means TNT must be used to produce mana. Other sources of explosions
like End Crystals or Creeper Statues won't work.

\subsubsection{Fuel}
Entropynniums can only eat Primed TNT entities and produce 6500 mana for
each. Importantly, the fuel is \it{not} TNT, it's Primed TNT entities. This
means TNT dupers can be used to fuel the entropynnium at no cost.
.
\section{Spreaders}
Spreaders are arguably the most important part of mana generation. They're the
only method of getting mana out of flowers and are generally the bottleneck for
maximizing generation rates. For a brief overview, spreaders work by shooting a
single mana burst at a target. The burst has to physically move to the target
until it collides and the spreader can shoot another.

Since a spreader's limit is the fact it must wait for it's burst to reach the
target before 
\end{document}
